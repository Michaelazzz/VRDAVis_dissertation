\begin{description}

  \mich Do you have any thoughts? On the experience or just any feedback?

  \alex I mean, yeah, I guess. am I allowed to put on my i-Davie hat.

  \mich It's actually quite beneficial that you compare it to i-Davie?

  \alex Well, I think one of the things that people notice right away with i-Davie. I can't speak for other people. But when it comes to the, I guess what with this kind of software, it comes down to the visualisation and plus the interactivity. So this feels like it's really nice that I can actually I get that self contained standalone visualisation. That's awesome. It just the interactivity, I feel it's not there. Where it's like, yeah, I can draw the box, the box felt a little bit funny. Like, I didn't feel like it was like one-to-one, like going through the cube. Like, I don't know if it was growing from the middle or? Yeah, whichever people have requested for i-Davie, but I don't like that. I think it's, it can be convenient for certain things. But others, I think it's actually off-putting because it's almost like I don't know, I mean, because I've always used like, Photoshop and things like that. Be aware, I'll never have a situation where I click and hold and then uh, then the selection box gets bigger and smaller, right? I just I select this, and I still and it's like one to one what I want to have selected, right. And it's maybe a little bit inconvenient for us. I want to select around maybe an object of interest, which astronomers do. And the thing to kind of notice. And then it kind of goes. And then that being that, maybe that could be a separate mode, I don't know. That's not the only interactive kind of element that makes it feel bad, that the cube seems kind of distant and I'm not actually interacting with it. And I think that's especially the case where I don't actually have a cursor on my hand to show like, Okay, this is the value of that point. There's value in that point. So I think that would definitely be kind of a earliest add on, I guess to this is adding the real time kind of interaction of like, this is the value of that point. And that is the what he called the coordinate to that point. And then I can move it through the cube and then actually find where I'm at because otherwise it's a little bit yeah, even though it is the 3D space. I don't feel like the the the the immersion is kind of gone. The presence, I guess is also kind of gone.

  \mich That drag from the middle function of the selection box was just the easiest to implement at the time.

  \alex That makes sense to use. Because it probably for in the case of astronomy is probably a lot more useful. Because usually astronomers are worried about individual spots and they want to select that region and then just get a designated kind of radius around that to crop to it. That makes sense. But, yeah, and then the menus did feel a little bit funny. Yeah, because I don't know what. Yeah, when you're dealing with all these, like very low level kind of libraries, then you need to add into your seeing, versus like us where we deal with Unity, where a lot of things are built for us. And it's definitely a challenge. But I think with menus, it comes down to a lot of the time, too, if you can emulate what people are used to. So when I want to move a menu, in, in Windows, I need to grip the side of the mouse and move it around, I just literally click the ribbon at the top and drag it around, right. And in VR, that makes sense for me as well. You just have to maybe make it a little bit fatter, that banner at the top just to make sure they can drag it around. Be I think other than that, it's it's stunning. To me, the main thing is that is the volumetric rendering that you got in on the standalone device. It's incredible. And then the performance didn't seem that bad, it look like didn't stutter at all.

  \mich The stutter was awful. Like when I took him to Taiwan, it was so bad because the UI library I was using was running additional loops, like every frame. So it was just making the computation of the scene just astronomical. As soon as I took that library out it ran so smoothly. So then that's that's why they have janky menu, because that's what is the trade off for performance.

  \alex And that's not a menu said like a UI system that's made for VR.

  \mich No it's not specifically made for VR, it's just canvases on plains, but I did manage to get chartJS onto one of those. So that's a dynamic chart that changes with the data. Which I'm also very happy with.

  \alex Yeah, I mean, at the end of the day, it's it's quite the task you add without using Unity. Because for us, it was just dragging things in and they were kind of tested to work already versus doing a lot of things from scratch on your end. They I think it's certainly a good, a good start.

  \mich What do you think of the workflow integration? Successful? How do you feel about it?

  \alex So between the desktop?

  \mich Yeah, so if you had to start with the desktop, and then move on to the VR side of the headset? Do you find that if the astronomers want to look at a cube in VR, they typically just start from the VR?

  \alex You know, I mean, I definitely would think that like a similar system, where I'm selecting the file from the menu on the desktop, and then going into VR to do what VR is good for. And that's this to see things in 3D. That's to see things in 3D, navigate around the cube. I don't want to have to select a cube from within VR unless necessary, because sometimes, maybe I just made one little mistake. Maybe I'm setting parameters on the cube or something and I want to go back really quick. There wouldn't be or maybe there was a second cube. And it'd be nice to load up really quick. But that's not worth putting a whole file selection. And I think basically keep VR to a VR is good at and that especially that's in both directions. So basically sending things, though, so before I put on the VR, and then after I take off the headset, what am I getting out of it? I mean, there I didn't. I didn't feel like I was producing anything yet. And it's difficult because I'm seeing stats, I'm seeing the graph. And maybe downside point to an area. But I mean a lot of these numbers. I do have a paddock and write it down or something. So I don't know if I'm exporting or, or putting, like what actually productive I'm getting out of the session, I feel because there's not a other direction thing yet. And that's kind of lacking. So in terms of being in the workflow, it's more of a work endpoint. But yeah, it's good that I can select the file from the desktop and then go in. And then I just have it as well. It's really nice that when things did crash, it was really straightforward just to do just to refresh and then that was actually quite nice. That things didn't just like it catastrophic. And then you have to ask you who's the expert at it. To grab the cube then fix it for me like it actually was quite straightforward. Just reload the scene.

  \mich Okay, that's really nice. I was like, yeah, it's just a lot of like, polish stuff. Additions like next steps. Yeah, that's, that's what I'm getting from the feedback. So I'm very happy because I'm like, Oh, this is this is garbage Oh, that's great feedback. Okay. Thank you so much. That's great feedback.

\end{description}