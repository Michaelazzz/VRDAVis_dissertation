\begin{description}
    \mich Now we are just going to move onto some free-form questions about what you experience. Okay so obviously you've worked with VR astronomy data visualisation software before. Do you have any feedback on the experience as a whole?
    
    \omri I mean very impressive, I think i was saying the hardest part for me was turning the data. Aside from that it seems its quite clear to see what's in the cube and I did find also you know when the cube goes behind the menu then one has to find a way to move around that.
    
    \mich Yeah so some user interface issues which are common.
    
    \omri And then I guess the fact that the crop cube sticks around.
    
    \mich User interface issues, completely understandable. And then what do you think of the workflow, of like having a cube open in CARTA you just go to the VR headset, do whatever you need to do in VR and then just go back to the computer when you're done and all the data from VR is back on the computer. How do you feel about that workflow?
    
    \omri That sounds like that would be great, because I mean CARTA to have the slices in CARTA and just looking at it like that it's sometimes hard to get a good visualisation of what the cube is doing.
    
    \mich It seems like you've had a very positive experience with VRDAVis. Most of the time these post-test interviews determine what the major issues are. I see you didn't find any major issues with the system itself. It's value proposition is still there even with a clunky interface.
    
    \omri I think it would be a very helpful thing to have. As a way to have a quick look in VR for just regular astronomy use.
    
    \mich And then one last question, since you have used i-Davie before how do you think the two systems compare?
    
    \omri I mean, so as I say just general interface, turning and manoeuvring the cube is more streamlined in i-Davie and I mean I think also the general clarity of the data in the cube. I don't know much about the technicalities of how that works but it did seem visually clearer on i-Davie, I don't know why that would be.
    
    \mich The reason for that is i-Davie run on a higher powered machine compared to this headset, so it can load in more data, so the resolution of the cube you are looking at looks a lot nicer. This tries to counteract having to load in so much data for performance by loading in pieces. This is why the crop functionality is a core feature. So like as you drill down the data becomes higher resolution.
    
    \omri I mean, I think for the purpose of having a quick look at something instead of having to load it up in a whole i-Davie, it would serve its purpose.
\end{description}