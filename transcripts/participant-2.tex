\begin{description}
    \lucia That was quick. Yeah, I think yeah. If you because of the of the selection, that is a shape that is not attached to the cube. It's also because I'm so familiar with how i-Davie works, right? You have all these other things. So it's actually what I expect to do I guess. Yeah. So so maybe that's why I found it a little bit counter-intuitive, but it's also because I'm used to the mapping already. But it's working in the transfer minimum data search its a matter of perfecting it a little but there is something there.
    
    \mich It does require more polish, I do completely understand what you mean by the controls. They are a little weird, and more thought needs to be put into it which will be a done though the iterative development process.
    
    \lucia But I didn't have any problem with motion sickness, maybe I want it to be forgiven with respect to...there was not a lot of jitter. So I guess it just didn't do it that it didn't like go crazy.
    
    \mich At a stage it was jittering to the point where I was getting sick. So that was that was because of the UI library we were using was adding extra rendering frames within the rendering loop. each frame has a rendering loop. So every rendering function it was adding more loops inside the render function. So it's just crazy, but we counteract the jittering by scaling down the steps it takes with the raycast shader. So the loop would just stop. We just tried to keep it as smooth as possible, because it is a comparatively lower powered hardware compared to the computer you use to run i-Davie on.
    
    \lucia So but then the you're, you're using the headset computer to do some of this calculation, or all of them.
    
    \mich Yeah, all of them. An I just wanted to ask you some questions. It's just to make sure I don't forget anything. So obviously the struggles you encountered was with actual interacting with the cube.
    
    \lucia Yeah, yeah. But as I said, because I'm used to interacting with the cube in a different way. Also, my, my way of reaching to the wrong bottom and need to remember the now okay, this is the trigger. And this is the other one says different words. 
    
    \mich Yeah, its just a learning curve.
    
    \lucia Yeah. But it's not difficult to you know, if I wasn't used to the other one, I would probably just do it. It's just that I'm used to do something else.
    
    \mich You know, obviously, you've seen data like this being displayed in this way in VR. So yeah, what do you think about?
    
    \lucia Well, I think our right marching is better to see the definition of the data. This one, it's really, almost, I can't wait. You know, when you see those sci-fi movies of the 70s, where there are the space travel, things of those lines appearing and disappearing in that world, believe me, as I perceive that vision, so you can't really see much of the data details. So that's, that is a little bit limiting in that respect. But again, I'm so used to see how we do it a different way that I guess I'm a little bit less forgiving than others, maybe you know.
    
    \mich But that's absolutely fine. What did you think of this, this workflow, moving from the laptop, to the headset?
    
    \lucia I'm thinking it would work? I mean, I think I'm a little bit biased because I work in this project (i-Davie). But, yeah, I think because of the size of the data we will have to deal with, I think its going to be important to have capability of reducing the data into the data of interest before you immerse yourself. This is of course with the idea that people will try visualise a cube of an entire survey in one go and go to the single source to analyse. Depends really on what type of science...of pipeline people will use. Which is a new thing. We don't know really because we've never had the capacity to do that. So I don't know. Having both sides of the thing, because VR does not good to do everything perfectly because it's so different to the screen.
    
    \mich I don't thing the system displayed its capabilities of being able to drill down into the cube. It's a little bit funny. A little but temperamental. Sometimes you get a good piece of the data. Which is something I would like to correct. An just one more, how does it compare to i-Davie?
    
    \lucia Well, how tricky, I mean, I really such a mature thing right now. I mean, you can see this is still experimental. I like it better how we grabbing things in the i-Davie, to be honest, I find the more intuitive or we have put ourselves into understanding how people and also maybe now I decide I'm so used to that, that I find it easier. So the grabbing of this, this action of grabbing space, instead of the trigger, I find it more intuitive because it's really grounding in the concept. So for example, the mapping of the control if you want the rendering also, to be clear, in i-Davie because of that, and of course, I mean, i-Davie has all the analytics. But I think the i-Davie problem is dealing with big data, this algorithm that works, you can have the marriage of the two. That's the idea I guess

\end{description}