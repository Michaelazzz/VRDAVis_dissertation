\section{Introduction}
\label{sec:intro}

% topic - what does the reader need to know to understand
% focus and scope - what specific aspect of the topic will be addressed
% relevance of your research - how does the work fit into existing research, what is the problem
% questions and objective - what does your research aim to find out
% an overview of the structure

Radio astronomy research involves the processing and exploration of enormous amounts of data.
The quantity of data available for research increases as data collection instruments improve in quality.
Modern radio telescopes are equipped with larger and more sophisticated sensors, generate data ranging from terabytes to petabytes each day.
To extract knowledge from this data, astronomers need comprehensive visualisations, and analytical tools that enable real-time interaction and exploration.
It is a major challenge to visualise such large datasets and involving issues related to data storage, querying, visual presentation, interaction, and personalisation~\cite{Bikakis2018}.

Astronomy visualisation systems conventionally visualise the three-dimensional datasets using two-dimensional media, such as screens.
Ideally, the data should be presented in a way that maintains its dimensional integrity which allows users to interact with it as if it existed in the real world.
A promising solution lies in the use of virtual reality (VR) headsets, which can display three-dimensional data in three-dimensional space.
VR headsets use stereoscopic displays and motion tracking to provide a more intuitive and interactive environment for data exploration.
However, they are limited by comparatively low computational power comparable to a smartphone and cannot process and visualise the massive data cubes generated by radio telescopes.

An ideal visualisation system must be able to handle very large datasets and operate on machines with limited computational and memory resources such as laptops and virtual reality headsets.
% This is particularly difficult because these comparatively lower-powered systems must be able to accommodate these aforementioned requirements.
% Astronomy is not the only field which has encountered these problems, there have been applications created in the medical and architectural fields, as well as several experiments in biology, geology, meteorology~\cite{Ferrand2016}.

\subsection{Objective}
This research aims to explore the challenges of astronomy data cube visualisation within an interactive environment as these cubes approach gargantuan sizes.
It implements a prototype system which can visualise very large volumetric data cubes within a VR environment, by using an intelligent strategy which can scale to handle ever-increasing sizes without sacrificing system performance or user experience.
The research will compare the performance of the prototype system to a system with similar functionality, while also taking user feedback into account in a qualitative evaluation. 
It will evaluate the validity of the chosen system design with the aim of contributing to the understanding of volumetric visualisation tools and their development and implementation.


\subsection{Research Questions}
% explain the questions
The first question investigates whether the prototype visualisation system can be implemented as a remote application.
This system design prioritises the user experience, ensuring that the prototype system can be adopted into the research workflow with the least amount of friction.

The second question examines the system's capability to handle very large data cubes. 
The system would scale with incredibly large data cubes, several times larger than the memory of a lower-powered device.
These lowered powered devices are devices such as laptop or VR headset.
% a indication of success is that its performance must remain stable as the data-cube it visualises increases in scale

Finally, the research will compare the prototype visualisation system to a system which uses a co-located computer, such as i-Davie.
This comparison will involve user testing, where participants will evaluate the VRDAVis system and provide feedback on its performance, usability, and overall experience.
% to compare how they perform from a user perspective
\begin{enumerate}
    % device be implemented as a remote visualisation system?
    \item Can the proposed system architecture be implemented as a remote application for astronomy data visualisation system? 
    % not fit inside the memory of a VR device or desktop computer?
    \item Can the system produce performance metrics within certain thresholds when presented with a data cube several times the size of the device's memory?
    % such as the Oculus Quest 2, without having a co-located computer take on the majority of the computational workload?
    \item How does the proposed system which uses a standalone virtual reality (VR) headset client, such as the Oculus Quest 2, compare to a system using a co-located computer to take on the majority of the computational workload?
\end{enumerate}

\subsection{Overview}
% nice skeleton but needs a few details
The document is structured into five main sections:
\begin{enumerate}
    \item An introduction that includes the objectives and goals of the research. 
    \item A literature review which explores existing research, and covers topics such as virtual reality, visual analytics, data visualisation, and graphics. 
    \item A system design and implementation section describes the construction of the system. 
    \item An initial evaluation of the system, which assesses the performance of the system. 
    \item The conclusion summarises the research and outlines areas for improvement.
\end{enumerate}

% flesh out
Section 2 - literature review
Section 3 - system design
Section 4 - initial evaluation
Section 5 - evaluation results
Section 6 - conclusion