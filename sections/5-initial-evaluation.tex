\section{Initial Evaluation}
\label{sec:initial-evaluation}

% have industry experts review the system and log their feedback to come to a conclusion

% intro
%   explain scope
%   this evaluation is designed to judge the sentiment of industry experts
%   open a line of communication with potential users 
%   receive feedback and guidence on future improvements
%   the system is optimised for user experience on the standalone VR headset, therefore, their feedback is crucial
%   semi-structured interviews guides the participants through their interactions with VRDAVis and allow them to progress at their own pace
%   some questions will be asked once they are finished with their evaluations
The evaluation was designed to judge the sentiment of expert users and open a line of communication to receive feedback and guidance on future improvements. 
These expert users were individuals who are researchers in the astronomy field; they also required some experience with VR and experience of using a similar system such as i-Davie. 
They are aware of the requirements of a successful VR experience, as well as a successful astronomy data visualiser.
VRDAVis is an experimental system and has some shortcomings that may confuse a novice user. 
The VRDAVis system is optimised for user experience on the standalone VR headset, which makes their feedback crucial. 
Semi-structured interviews guided the participants through their interactions with VRDAVis, allowing them to progress at their own pace. 
Additionally, some questions were asked once they had finished their evaluations to further refine the system based on their insights and experiences.

% methodology
% the design of the experiment
%   the user will be asked to use the system from starting from the laptop and themoving to the vr headset
%   the user will be asked to complete tasks such as
%       select a file to view on the laptop
%       transfer the state to the VR headset
%       enter the VR environment via the VR headset
%       manipulate the visualiasation of the file
%           this invloves translating, rotating, and scaling the visualisation
%       crop the visualisation 
%   all the participants use the same system
In the experiment, the participant will be asked to start using the system on a laptop and then move to the VR headset. 
The participant will be required to complete tasks such as selecting a file to view on the laptop, transferring the state to the VR headset, entering the VR environment via the VR headset, and manipulating the visualisation of the file. 
This manipulation involves translating, rotating, and scaling the visualisation, as well as performing an action where the visualisation's resolution is scaled up as it is cropped. 
All participants will use the same system to ensure consistency in the experiment.

\subsection{Participants}
% participants
%   the participants were pre selected from member if the Instituite of Data-intensive astronomy
%   all paticipants are briefed on tasks they will perform as well as the protocol if they start feeling nauseous while using the VR headset, this is known as simulator sickness
%   they are all asked to sign a consent form belfore the comensment of the evaluation
%   the participants evaluating the VRDAVis system are industry experts in astronomy and computer science
%   these participants are also well versed in VR technology
%   they are regular users of VR interfaces
%   also have experience using VR systems designed to visualise astronomy data
%   such as i-Davie
%   VRDAVis is an experimental system and has some shortcomings which may confuse a novice user
%   having some experience in VR bolsers the user from becoming disorientated and ahelps them avoid simulator sickness
%   during the duration of the evaluation if a user does begin to experience simulator sickness the evaluation is stopped to tend to the participant
%   once the participants recovers from the simulator sickness they can their continue with the evaluation or they can stop the evaluation

The participants in the experiment were pre-selected from members of the Institute of Data-Intensive Astronomy. 
All participants were briefed on the tasks they would perform as well as the protocol to follow if they start feeling nauseous while using the VR headset.
% This phenomena is a condition known as simulator sickness, and is common occurrence in users not well acquainted with the sensations that occur while using a VR headset.
% No long term harm comes from this as the feelings of nausea pass once the VR headset is removed.
All participants were all asked to sign a consent form before the commencement of the evaluation. 
It reiterates the tasks they will performing and what data will be collected during the experiment.

\subsection{System Set-up Description}
% System set-up description
%   a system used to visualise astronomy data cubes
%   a laptop and VR headset which communicate with remotes systems as well as each other
%  all for the purpose of visualisaing an astronomy data cube on laptop or VR headset
%   full description of the VRDAVis system can be found in the system design section
To conduct the evaluation, a carefully designed set-up is essential to ensure smooth transitions between the laptop and VR headset, as well as to capture the participants' interactions and feedback accurately. 
The environment is a quiet, controlled space free from external disturbances, equipped with chairs and adequate lighting. 
The equipment includes a laptop with the necessary software for viewing and manipulating the selected file, along with a standalone VR headset and the required controllers. 
Participants will begin by selecting and viewing the file on the laptop, then transition to the VR headset and interact with the visualisation.
These actions include translating, rotating, scaling, and cropping the visualisation. 
The procedure involves a briefing area where participants are informed about tasks and protocols, including handling simulator sickness, and sign consent forms. 
The task area houses the laptop, while a designated transition spot allows for easy movement to the spacious VR area. 
Monitoring and support are provided to the participants by the conductor of the experiment, who observes for signs of discomfort and offer immediate assistance, with video and audio recording equipment capturing interactions and feedback. 
Post-evaluation, participants relax in the briefing area where they discuss their experience and answer some questions from the conductor. 
Clear guidelines are in place for handling simulator sickness, including the option to stop and resume the evaluation. 
This set-up ensures a controlled, supportive environment, prioritising participant well-being and enabling the collection of detailed feedback on the VRDAVis system.

% metrics / questions
%   gather qualitative data about the user's experience
%       anything the user highlights during their time using the system
%   see which taks they struggle with the most
%   how well they interact with the user interface
%   what they think of seeing the data-cube visualised in three-dimensional space
%   what they think of interacting with the data-cube visualised in three-dimensional space
%   what they think of the workflow of the system
\subsection{Metrics}
During the evaluation, several metrics were collected to assess the VRDAVis system. 
Qualitative data about the user's experience was gathered, including any highlights or noteworthy comments they made while using the system. 
Observations were made to identify which tasks participants struggled with the most and how well they interacted with the user interface. 
Feedback was collected on their impressions of seeing the data cube visualised in three-dimensional space and their thoughts on interacting with it. 
Participants were asked for their opinions on the overall workflow of the system, providing valuable insights into its usability and effectiveness.
