\section{Evaluation Results}
\label{sec:evaluation-results}

% introduction
The goal of the evaluation was to gain insight and feedback from users regarding the VRDAVis system. 
Feedback was gathered by having users evaluate the VRDAVis system, and the results highlight both the successful parts and areas for improvement.

% PARTICIPANT 1 - ALEX
%   developer of i-Davie

% ### Feedback Summary

% #### Participant's Perspective on i-Davie vs. Current Software

% 1. **Visualization and Interactivity**
%    - **Strengths:**
%      - Standalone visualization is appreciated.
%      - Volumetric rendering on a standalone device is seen as a significant achievement.
%      - Performance is commendable; it runs smoothly without noticeable stutter.
%    - **Weaknesses:**
%      - Lack of interactivity: The participant finds the interactive elements, such as the selection box, not very intuitive
%      - Immersion: The 3D space does not feel immersive, partly due to the absence of a cursor to show point values and coordinates.
%      - Menu System: Menus feel awkward and are not optimised for VR, affecting usability.

% 2. **Comparison with i-Davie**
%    - **Differences:**
%      - i-Davie is praised for its interactivity, which is perceived as lacking in the current software.
%      - Selection mechanism: i-Davie allows for precise selection, which the current software does not replicate well.
%    - **Suggestions:**
%      - Implement real-time interaction features to show values and coordinates.
%      - Improve the selection mechanism to be more intuitive and precise.

% 3. **Menu and UI System**
%    - **Issues:**
%      - The current menu system feels off and not user-friendly.
%      - UI libraries are not designed specifically for VR, leading to performance trade-offs.
%    - **Suggestions:**
%      - Emulate familiar UI interactions from desktop environments (e.g., dragging menus by the top banner).
%      - Make UI elements more VR-friendly and responsive.

% 4. **Workflow Integration**
%    - **Process:**
%      - Participants prefer starting with the desktop to select files and then transitioning to VR for 3D visualisation.
%      - Direct file selection in VR is less preferred due to occasional need for quick adjustments.
%    - **Feedback:**
%      - Integration is seen as more of a workflow endpoint rather than a seamless process.
%      - Lack of productivity features: Users need a way to export or record data from VR sessions effectively.
%    - **Suggestions:**
%      - Enhance the workflow by enabling productive output from VR sessions.
%      - Ensure straightforward crash recovery to prevent workflow disruptions.

% #### Interviewer's Response

% - **Acknowledgment:**
%   - Recognised the feedback about interactivity, menu usability, and workflow integration.
%   - Acknowledged the limitations due to the current UI library and its performance impact.
% - **Future Improvements:**
%   - Plans to address interactive elements and menu usability in subsequent updates.
%   - Intends to focus on polishing and adding necessary features based on the feedback received.

Participant 1 noted a positive experience with the visual quality and interactive aspects of VRDAVis, specifically praising the ease of use for cropping and analysing different data segments. 
They highlighted the significance of having an immersive experience for visualising complex data, which they found to be more intuitive compared to traditional two-dimensional methods. 
However, they mentioned the importance of improving the system’s stability, as they experienced some technical glitches. 
They also suggested that additional features, such as more sophisticated data manipulation tools and enhanced resolution, could further enhance the usability and effectiveness of VRDAVis. 

% leave out immediate summaries
% The feedback highlights key areas for improvement, including interactivity, UI/UX design, and workflow integration. 
% The participant appreciates the strengths in visualisation and performance but suggests enhancing the interactive experience and refining the menu system for better usability in VR. 
% They concluded that while VRDAVis shows promise, it would benefit from iterative development and refinement to fully realise its potential in complementing existing tools like i-Davie.

% PARTICIPANT 2 - Lucia
% ### Feedback Summary

% #### Participant's Perspective on i-Davie vs. Current Software

% 1. **Selection Mechanism and Interaction**
%    - **Challenges:**
%      - The current selection mechanism is not intuitive compared to i-Davie.
%      - Participant's familiarity with i-Davie influences their expectations and interaction comfort.
%    - **Suggestions:**
%      - Refine the selection shape to be more intuitive and user-friendly.
%      - Adjust controls to match more familiar schemes or provide a customizable option for users.

% 2. **Motion and Performance**
%    - **Strengths:**
%      - The participant did not experience motion sickness or jitter, indicating stable performance.
%    - **Weaknesses:**
%      - Historical issues with jitter due to the UI library, which have since been mitigated.
%    - **Suggestions:**
%      - Continue optimizing performance to prevent any motion sickness and improve stability.

% 3. **Computational Load and Hardware**
%    - **Observations:**
%      - Current software relies on the headset's computational power, unlike i-Davie which utilizes a more powerful desktop setup.
%    - **Feedback:**
%      - Acknowledge the difference in hardware capabilities and optimize processes accordingly.

% 4. **Data Visualization Quality**
%    - **Issues:**
%      - The current software lacks the definition and detail in data visualization compared to i-Davie.
%      - Participant likened the current visual experience to outdated sci-fi depictions, indicating a need for improved clarity and detail.
%    - **Suggestions:**
%      - Enhance the rendering quality to provide more detailed and clear visualizations.
%      - Incorporate better algorithms for data display to match or exceed i-Davie’s capabilities.

% 5. **Workflow Integration**
%    - **Process:**
%      - Participant finds the workflow from laptop to headset acceptable but feels biased due to their involvement with i-Davie.
%    - **Challenges:**
%      - Handling large datasets: It’s crucial to reduce data before immersion in VR due to size and complexity.
%    - **Suggestions:**
%      - Develop capabilities for data reduction and filtering before VR visualisation.
%      - Consider how different types of scientific analysis might necessitate various workflow adjustments.

% 6. **Comparison with i-Davie**
%    - **Observations:**
%      - i-Davie is more mature and intuitive, especially in terms of interaction and data rendering.
%      - Current software feels experimental and less refined in comparison.
%    - **Strengths of i-Davie:**
%      - Intuitive grabbing actions and control mapping.
%      - Better rendering clarity and comprehensive analytics.
%    - **Suggestions for Improvement:**
%      - Aim to merge the strengths of both systems: use i-Davie’s intuitive interaction and analytics with the innovative aspects of the current software.
%      - Focus on perfecting user interactions and rendering details.

% #### Interviewer's Response

% - **Acknowledgment:**
%   - Recognized the feedback on selection mechanism, interaction comfort, and the need for more detailed visualization.
%   - Addressed the historical jitter issue and explained the steps taken to mitigate it.
% - **Future Improvements:**
%   - Plans to refine the interaction controls and improve the data visualization quality.
%   - Continue optimizing performance and consider the differences in computational capabilities between headset and desktop setups.

Participant 2 found the initial experience counter-intuitive, primarily because they were accustomed to the control mapping of i-Davie. 
% They noted some difficulty with the controls but acknowledged that this was partly due to their familiarity with a different control mapping. 
Despite the learning curve, they did not experience motion sickness and appreciated the smoothness of the system on the VR headset. 
They highlighted the need for better visual definition and clearer data details. 
The participant emphasised the importance of being able to reduce data before immersion in VR and noted that VRDAVis and i-Davie have complementary strengths, suggesting a potential marriage of the two systems for handling big data effectively.
% leave out immediate summaries
% The feedback underscores the need to improve the selection mechanism, enhance data visualisation quality, and refine user interactions. 
% They emphasised the importance of intuitive controls and detailed rendering seen in i-Davie.

% PARTICIPANT 3 - Sushma
% Workflow with i-Davie - takes very long to set-up and load the cube
% VRDAVis is very fast to set-up an use compared to i-Davie
% more features
% more refined features

% ### Feedback Summary

% #### Participant's Perspective on Current Software vs. i-Davie

% 1. **Headset Experience**
%    - **Challenges:**
%      - Initial discomfort with the headset, which improves with usage.
%    - **Feedback:**
%      - No major issues, just a matter of getting used to the device over time.

% 2. **Feature Set and Functionality**
%    - **Weaknesses:**
%      - Current software lacks certain features, especially related to contrast adjustments.
%      - Difficulty in seeing emission lines due to inadequate contrast.
%    - **Suggestions:**
%      - Enhance contrast features to improve visibility of high and low-density features.
%      - Add options to adjust the scale, including maximum and minimum values, to tailor the view based on the features of interest.

% 3. **Comparison with i-Davie**
%    - **Observations:**
%      - i-Davie is more mature and feature-rich, especially in handling larger data cubes.
%      - Current software has the advantage of being accessible from any computer, which is highly appreciated.
%    - **Strengths of Current Software:**
%      - Accessibility from anywhere without needing a specific machine.

% 4. **Visualization Quality**
%    - **Feedback:**
%      - Visual quality is generally good but can be improved with better contrast adjustments.
%      - Important to visualize different density features and backgrounds more effectively.
%    - **Suggestions:**
%      - Improve the visualization by allowing finer control over contrast and scaling settings to highlight specific data features.

% #### Interviewer's Response

% - **Acknowledgment:**
%   - Recognized the feedback regarding headset comfort and the need for more features.
%   - Agreed that the current software is less mature than i-Davie but appreciated the feedback on accessibility.
% - **Future Improvements:**
%   - Plans to work on enhancing contrast features and adding options for adjusting scale values.
%   - Acknowledged the importance of continuing to iterate and add more features to match user needs.

Participant 3 did not encounter significant struggles with the headset, noting that familiarity would improve with use. 
They suggested that the system could benefit from enhanced contrast to improve the readability of the visualisation. 
Comparing VRDAVis to i-Davie, they found i-Davie to be more feature-rich but appreciated VRDAVis’s ease of accessibility. 
% They indicated that better contrast would improve visualisation of high-density and low-density features. 
% leave out immediate summaries
% Their feedback highlights the importance of enhancing contrast and visualisation features to better distinguish data features in high and low densities. 
% The comparison with i-Davie underscores the maturity and feature richness of i-Davie, while appreciating the accessibility of VRDAVis.
% Overall, they found the system’s workflow between laptop and VR headset to be effective and valued the capability to adjust visual parameters based on their needs.

% PARTICIPANT 4 - Omri
% astronomers what a quick and easy way of viewing the cube in VR

% ### Feedback Summary

% #### Participant's Perspective on Current Software vs. i-Davie

% 1. **Overall Experience**
%    - **Strengths:**
%      - The participant found the experience impressive and effective for visualizing data cubes.
%    - **Weaknesses:**
%      - Difficulty in turning and maneuvering the data cube.
%      - User interface issues, such as the cube going behind the menu and the crop cube persisting.

% 2. **User Interface**
%    - **Challenges:**
%      - Maneuvering the data cube and the interface could be more streamlined.
%      - Occasional issues with the cube overlapping with menus.
%    - **Suggestions:**
%      - Improve the user interface to make navigation and interaction more intuitive.
%      - Address issues with menu and cube overlapping.

% 3. **Workflow Integration**
%    - **Feedback:**
%      - Positive response to the workflow of integrating CARTA with VR.
%      - Appreciated the ability to visualize cubes in VR, which is sometimes difficult in CARTA alone.
%    - **Suggestions:**
%      - Continue developing the integration to enhance the seamless transition between desktop and VR environments.

% 4. **Comparison with i-Davie**
%    - **Observations:**
%      - i-Davie offers a more streamlined and clearer interface for maneuvering and visualizing the data cube.
%      - Visual clarity is higher in i-Davie due to the higher-powered hardware it runs on.
%    - **Strengths of Current Software:**
%      - Quick and convenient for a preliminary look at the data without needing a high-powered machine.
%    - **Weaknesses Compared to i-Davie:**
%      - Lower resolution and visual clarity due to performance limitations of the headset.
%      - Maneuvering and interface not as smooth and intuitive as i-Davie.

% 5. **Performance and Resolution**
%    - **Feedback:**
%      - Acknowledged the performance trade-off in the current software due to hardware limitations.
%      - Understood the importance of the crop functionality for improving resolution as one drills down into the data.
%    - **Suggestions:**
%      - Continue optimizing the resolution and performance balance, possibly by enhancing the crop functionality and data loading techniques.

% #### Interviewer's Response

% - **Acknowledgment:**
%   - Recognized the feedback on maneuvering difficulties and user interface issues.
%   - Explained the performance and resolution trade-offs due to hardware limitations compared to i-Davie.
% - **Future Improvements:**
%   - Plan to refine the user interface to make it more intuitive and address overlapping issues.
%   - Continue working on optimizing performance to improve visual clarity and interaction smoothness.

Participant 4 was impressed by the system but faced difficulties in rotating the data and managing the interface when the cube was ocasionally obstructed by the menu. 
They found the workflow from laptop to VR headset to be advantageous, particularly for its ability to provide a more immersive visualisation. 
They noted that while VRDAVis still had some user interface issues, it had the potential to be a helpful tool for quick looks at data. 
Comparing VRDAVis to i-Davie, they found i-Davie to be more streamlined in terms of manoeuvring and visual clarity, which is likely due to i-Davie using a specialised standalone computer to run on.

% leave out immediate summaries
% Their feedback highlights the importance of improving the user interface and manoeuvring capabilities to match the intuitive experience of i-Davie. 
% The participant appreciated the current software's quick and convenient visualisation capabilities, despite the lower resolution due to hardware limitations.

% PARTICIPANT 5 - Nabeelah
% crop data envelops the user after new data is loaded 
% much easier quicker and easier to load and view the cube in VR

% ### Feedback Summary

% #### Participant's Perspective on Current Software vs. i-Davie

% 1. **Initial Struggles**
%    - **Challenges:**
%      - Initial difficulty in zooming in and out.
%      - Moving the menu away was initially tricky but manageable after some practice.

% 2. **Visualization Quality**
%    - **Feedback:**
%      - Found the visualization of the masked image to be great.
%      - Expressed interest in seeing what an unmasked image would look like for comparison.

% 3. **Comparison with i-Davie**
%    - **Interface:**
%      - The participant found VRDAVis to be simpler and quicker to get into compared to i-Davie.
%      - Recalled an issue with the FITS file in i-Davie that made it take about 10 minutes to open the cube, highlighting the ease of VRDAVis.
%    - **Functionality:**
%      - Mentioned not performing all actions with VRDAVis that were done with i-Davie but appreciated the simplicity.

% 4. **Integrated Workflow**
%    - **Feedback:**
%      - Appreciated the integrated workflow between the computer and the headset.
%      - Liked the flexibility of being able to put on the headset at any time to look at something.

% #### Interviewer's Response

% - **Acknowledgment:**
%   - Recognized the initial struggles with zooming and menu movement.
%   - Noted the participant's interest in comparing masked and unmasked images.
%   - Highlighted the ease of getting into VRDAVis compared to i-Davie.
%   - Appreciated the positive feedback on the integrated workflow.

Participant 5’s main struggle was initially handling the zoom and moving the menu but found these manageable over time. 
They appreciated the quality of the image used for visualisation.
They noted that VRDAVis’s interface was simpler to access compared to i-Davie, which they found cumbersome to open and load data. 
They valued the integrated workflow between the laptop and VR headset, highlighting its convenience for quickly checking data without needing an elaborate setup.
% leave out immediate summaries
% Their feedback underscores the user-friendly nature of VRDAVis, particularly in terms of ease of access compared to i-Davie. 
% Initial difficulties with zooming and menu navigation were noted but manageable with time. 
% The participant found the visualisation quality of masked images to be good and expressed interest in viewing unmasked images. 
% The integrated workflow was well-received, with the participant appreciating the flexibility and convenience it offered. 
% The feedback suggests that while VRDAVis is user-friendly and efficient, further enhancements in initial navigation and exploring different types of visualisations could improve the overall experience.

% PARTICIPANT 6 - Kyra

% ### Feedback Summary

% #### Participant's Perspective on Current Software vs. i-Davie

% 1. **Initial Struggles**
%    - **Challenges:**
%      - Accidental activation of the crop mode at the beginning of the evaluation.
%      - Interface errors, specifically clicking through the cube without realizing it, but found it manageable.

% 2. **Workflow**
%    - **Feedback:**
%      - Appreciated the workflow of starting from the laptop and transferring the state to the headset.
%      - Found this process quicker than loading the data directly into the headset, comparing it favorably to i-Davie which takes longer to load.

% 3. **Comparison with i-Davie**
%    - **Feedback:**
%      - Found VRDAVis and i-Davie to be quite similar in terms of functionality.
%      - Acknowledged that VRDAVis might be a basic version as it lacks some advanced features like moment maps and cropping cubes, which are present in i-Davie.

% 4. **Visualization Quality**
%    - **Feedback:**
%      - Described the visualization quality as good and comparable to i-Davie.
%      - Even with a low-quality cube, the participant could discern different intensities, emissions, and depth within the cube.

% #### Interviewer's Response

% - **Acknowledgment:**
%   - Recognized the initial struggle with the crop mode and interface errors.
%   - Noted the positive feedback on the workflow between the laptop and the headset.
%   - Acknowledged the comparison between VRDAVis and i-Davie, highlighting the similar functionality but noting the advanced features of i-Davie.
%   - Appreciated the feedback on the visualization quality, ensuring that even low-quality cubes provide discernible data.

Participant 6 encountered an issue with the crop mode activation at the beginning of the evaluation but found the problem minor. 
They appreciated the efficient workflow that allowed quick transitions between the laptop and VR headset, contrasting it favourably with i-Davie’s longer loading times. 
They found VRDAVis similar to i-Davie in terms of basic functionality, though they acknowledged i-Davie’s additional features like moment maps. 
They were satisfied with the quality of the visualisation, noting its ability to display different intensities and emissions clearly.

% Their feedback indicates a generally positive experience with VRDAVis, highlighting its user-friendly workflow and good visualisation quality, even with lower resolution data. 
% Initial struggles with interface errors were noted but found to be manageable. 
% The participant found the workflow of starting on a laptop and transferring to the headset to be efficient, comparing it favorably to the longer loading times of i-Davie. 
% While recognizing that VRDAVis might be a basic version lacking some advanced features of i-Davie, the participant found both systems to be quite similar in functionality. 
% The feedback suggests that VRDAVis provides a foundation for efficient and clear data visualisation, with potential for further enhancement in advanced features and interface usability.

% PARTICIPANT 7 - Henco

% ### Feedback Summary

% #### Participant's Perspective on Current Software vs. i-Davie

% 1. **Initial Struggles**
%    - **Challenges:**
%      - The crop button being in the line of sight, with the cube obstructing access to the button.
%      - Suggested solution was to move the cube to the side to access the crop button.

% 2. **Workflow**
%    - **Feedback:**
%      - Found the proposed workflow great and seamless once the cube is uploaded to the system.
%      - Highlighted the advantage of eliminating the "middle man" in the process of opening cubes, which would streamline their work.
%      - Suggested it would be beneficial if there was a way to take the headset off, use it for a different task, and then put it back on to manipulate and load different cubes.

% 3. **Comparison with i-Davie**
%    - **Feedback:**
%      - Participant mentioned they didn’t personally upload data to i-Davie, but usually had someone else do it.
%      - Appreciated the potential for independence in handling data with the VRDAVis system.

% 4. **Visualization Quality**
%    - **Feedback:**
%      - Described the visualization quality as great and crystal clear.
%      - No issues with jittering, although scaling was not a smooth motion.
%      - Found the data itself to be fine and clear.

% #### Interviewer's Response

% - **Acknowledgment:**
%   - Noted the initial struggle with the crop button and the participant's suggested solution.
%   - Appreciated the positive feedback on the seamless workflow and the independence it offers compared to i-Davie.
%   - Recognized the need for smoother scaling motion and took note of the participant’s overall satisfaction with the data visualization quality.

Participant 7 noted the placement of some buttons as a minor issue since they sometimes interfered with the data cube.
The primary issue was the crop button being obstructed by the cube, but the participant found this manageable by moving the cube. 
They found the workflow seamless once the cube was uploaded, appreciating the efficiency. 
Having used i-Davie before, they saw the potential benefits of VRDAVis in eliminating intermediary steps for data handling. 
They were satisfied with the visualisation quality, though they noted that scaling the cube was not always smooth. 
% The visualisation quality was found to be excellent, with no jittering, though the scaling motion could be smoother. 

% Their feedback indicates a positive experience with VRDAVis, highlighting its user-friendly workflow and clear visualisation quality. 
% The workflow was praised for its seamless nature once the cube is uploaded, and the participant appreciated the potential for greater independence in data handling compared to i-Davie. 
% Overall, the feedback suggests that VRDAVis provides a solid and efficient tool for data visualisation, with minor interface adjustments needed.

\subsubsection{Key findings}
% key points that have been compiled from all of the interviews

\paragraph{User Interface and Interaction}
Some users had some initial difficulty with controls, such as zooming, moving menus, and the unstable crop function.
Interface elements sometimes obstructed the view of the data cube and vice versa.
Some suggested adjustments made by the users to make the user interface more intuitive and less obstructive, possibly repositioning or redesigning the interface elements to enhance usability.

\paragraph{Workflow Efficiency}
The ability to transfer the workspace state from the laptop or desktop computer and to the VR headset was seen as quick and efficient by the users.
They also appreciated the seamless switch between the respective environments.
When the workflow was compared to i-Davie's, it was noted that i-Davie requires more time to load data cubes and has more complex steps.
VRDAVis was praised for its simpler and faster setup, which can be beneficial for quick analyses.

\paragraph{Visualisation Quality}
Generally, the quality of visualisation in VRDAVis was considered good.
The participants were able to discern different intensities and areas within the data cube.
It was found that while the visual quality was good, i-Davie’s specialised hardware provides better resolution.

\paragraph{System Capabilities and Features}
The current features of VRDAVis, such as cropping and visualising different data sections, were appreciated.
The system was seen as basic yet capable of handling regular astronomy tasks efficiently.
The users emphasised the need for more features for adjusting visual contrast, manipulating data cubes, and additional analytical tools similar to ones implemented in i-Davie.
% Ability to visualise unmasked data and compare it with masked data.
% Integration of advanced functionalities like those in i-Davie, including moment maps.

\paragraph{General Sentiments and Recommendations}
Participants found VRDAVis to be a promising tool, particularly for its speed and ease of use.
The system's ability to offer quick insights without extensive setup was also highly valued, and the prospect of greater independence and flexibility in data handling was considered a significant advantage.
The users also provided constructive feedback which suggested user interface adjustments, enhancements in visual quality and smoother interactions, more features, and improvements to the overall user experience.

In summary, while VRDAVis is seen as a user-friendly and efficient tool for data visualisation in VR, improvements in user interface design, available feature, and interaction smoothness can enhance its effectiveness and user satisfaction. 
The participants appreciated the system's potential to streamline workflows and provide quick, clear three-dimensional visualisations of astronomy data.
